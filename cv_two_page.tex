\documentclass[a4paper]{article}

\usepackage{tabularx}

\usepackage[margin=1in]{geometry}

\usepackage{ae}
\usepackage[T1]{fontenc}
\usepackage{CV}
\usepackage{hyperref}

\begin{document}

\pagestyle{empty}

%Ueberschrift
\begin{center}
\Large{\textsc{Curriculum Vitae}}
\vspace{\baselineskip}

\large{\textsc{Joseph P. McKenna}}
\end{center}
\vspace{1.5\baselineskip}

\section{Contact Information}
\begin{flushleft}
Florida State University\\
Department of Mathematics\\
1017 Academic Way\\
Room 208 Love Building\\
Tallahassee, FL 323036-4510\\
Phone: 443-995-7788 \\
Email: jmckenna@fsu.edu \\
Homepage: \url{www.math.fsu.edu/~jmckenna} \\
\end{flushleft}

%\section{Personal Details}
%\begin{flushleft}
%Gender: Male \\
%Date of birth: April 3, 1986 \\
%Place of birth: Annapolis, Maryland \\
%Citizenship: U.S.A.
%\end{flushleft}

\section{Education}
\begin{CV}
\item[08/12 - present] Ph.D. Mathematics (expected 12/06), Florida State University, Tallahassee, FL, GPA: 3.9, Advisor: Dr. Richard Bertram
\item[08/04 - 05/08] B.A. Mathematics, St. Mary's College of Maryland, St. Mary's City, MD. Major GPA: 3.5, Thesis: \emph{Regular Polytopes and Symmetry}
\item[09/09 -12/09] Graduate Non-Degree, University of Illinois, Chicago, IL, GPA: 4.0 
% \item[05/09 -12/09] Undergraduate Non-Degree, Harold Washington College, Chicago, IL, GPA: 4.0 
\end{CV}

%\section{Thesis}

\section{Teaching Experience}
\begin{CV}
\item [08/12 - present] Graduate Teaching Assistant, Florida State University
\begin{CV}
\item[08/15 - present] Foundations of Computational Math Teaching Assistant
\item[08/15 - 12/15] Applied Computational Math Instructor
\item[05/15 - 08/15] Calculus II Instructor
\item[01/14 - 05/15] Calculus I Instructor
\item[08/13 - 12/14] Precalculus Instructor
% \item[08/12 - 05/13] Business Calculus, Precalculus, College Algebra, Trigonometry, and Liberal Arts Mathematics Proctor
\end{CV}
\item[08/10 - 06/12] Junior High School Instructor, Peace Corps Ghana
\item[09/09 - 12/09] Tutor, Mathematical Science Learning Center, University of Illinois at Chicago
\item[09/07 - 05/08] Teaching Assistant, St. Mary's College of Maryland
\end{CV}

\section{Publications and Research Experience}
\begin{CV}

\item [12/15] J. P. McKenna, R. Dhumpa, N. Mukhitov, M. G. Roper, and R. Bertram, {\it Glucose Oscillations can Activate an Endogenous Oscillator in Pancreatic $\beta$-cells}, submitted.
\item [08/15] J. P. McKenna, J. Ha, M. J. Merrins, L. S. Satin, A. Sherman, and R. Bertram. {\it Calcium Effects on ATP Production and Consumption Have Key Regulatory Roles on Oscillatory Islet Activity}, Biophysical Journal, Vol. 110, Feb. 2016, 733-742. \href{http://dx.doi.org/10.1016/j.bpj.2015.11.3526}{doi}.
\item [08/15] M. J. Merrins, C. Poudel, J. P. McKenna, J. Ha, A. Sherman, R. Bertram, L. S. Satin, {\it Phase Analysis of Metabolic Oscillations and Membrane Potential in Pancreatic Islet $\beta$-cells}, Biophysical Journal, Vol. 110, Feb. 2016, 691-699. \href{http://dx.doi.org/10.1016/j.bpj.2015.12.029}{doi}.
\item [05/14 - 08/14] Summer Internship Program in Biomedical Research. Developed model of metabolic oscillations in pancreatic $\beta$-cells to elucidate biophysical basis of pulsatile insulin secretion. advisor: Dr. Arthur Sherman, National Institute of Diabetes and Digestive and Kidney Diseases, NIH
\item [10/11] J.McKenna, correct solution to {\it More and More Balls in Urns}, in American Mathematical Monthly, 118(8), October 2011, Problems and Solutions, pp 750-751, \href{10.4169/amer.math.monthly.118.08.747}{doi}.  
\item [4/11]J. McKenna, correct solution to {\it Permutations with specified left-to-right maxima}, in Mathematics Magazine, 84(2) April 2011, Problems and Solutions, pp 153-154, \href{10.4169/math.mag.84.2.150}{doi}.
\item [10/10] J. McKenna, correct solution to {\it Counting block fountains of coins}, in Mathematics Magazine, 83(4) Oct. 2010, Problems and Solutions, pp305, \href{10.4169/002557010X521877}{doi}.
\item [08/07 - 05/08] St. Mary's Project: Regular Polytopes and Symmetry.
% Authored exposition confirming fundamental results on existence of three- and four-dimensional polytopes, and describing geometrically rotation axis-planes of the four-dimensional simplex and hypercube. Authored in Java and deployed online, interactive visualization of polytopes undergoing rotations and reflections, adviser: Dr. Alex Meadows, St. Mary's College of Maryland
\end{CV}

% \section{Work Experience}
% \begin{CV}
% \item[07/11 - 06/12] Founding Manager, Atakora Junior High School Computer Center, Ghana. Established computer center in a renovated classroom as a venue that built teachers’ capacity to offer practical instruction and increased computer literacy among hundreds of students and community members.
% \item[11/10 - 06/12] Editor, Celebrate Languages Audio Project, Peace Corps Ghana. Wrote Java program that created language-learning lessons from interviews with Ghanaian language speakers.
% %\item[10/08 - 09/09] Receptionist, Hostelling International, Chicago, IL
% \item[11/08 - 03/09] Computer Assembler, FreeGeek, Chicago, IL. Assembled PCs operating Linux to offer low-cost computing to the economically disadvantaged.
% \end{CV}

\section{Presentations}
\begin{CV}
% \item[07/16] Minisymposium lecture (expected), {\it Glucose Oscillations Can Activate an Endogenous Oscillator in Pancreatic Islets}, Society for Industrial and Applied Mathematics Conference on the Life Sciences, Boston, MA
% \item[07/16] Contributed Poster (expected), {\it Reducing Condunctance-Based Models to Normal Form}, Society for Industrial and Applied Mathematics Conference on the Life Sciences, Boston, MA
% \item[02/16] {\it Reducing a Neuron Model to Normal Form at a Bogdanov-Takens Bifurcation}, Florida State University Biomathematics Journal Club
\item[01/16] {\it Glucose Oscillations Can Activate an Endogenous Oscillator in Pancreatic Islets}, Florida State University Biomathematics Seminar
\item[07/15] Minisymposium lecture, {\it Death and Reincarnation of the Dual Oscillator Model for Islet Oscillations}, Society for Mathematical Biology Annual Meeting, Atlanta, GA
% \item[11/15] {\it Invasion of Cancer Cells by Engineered Bacteria}, Florida State University Systems Biology Course
\item[05/15] Contributed Poster, {\it Rescuing the Dual Oscillator Model for Beta Cells from Inconvenient Data}, Midwest Islet Club Annual Meeting, Chicago, IL
% \item[04/15] {\it Classification of Bursting Mappings}, Florida State University Biomathematics Journal Club
% \item[01/15] {\it Bistability in Glycolysis Pathway as a Physiological Switch in Energy Metabolism}, Florida State University Biomathematics Journal Club
\item[08/14] Contributed Poster, {\it Mathematical Model of Metabolic Oscillations in Pancreatic Beta Cells}, NIH Summer Intern Poster Session
\item[01/14] {\it Chaos in the Hodgkin-Huxley Model}, Florida State University Biomathematics Journal Club
% \item[12/13] {\it Developing Models of Zebra Finch RA: Interneurons and Projection Neurons}, Florida State University Computational Neuroscience Course
% \item[09/13] {\it Glucose Metabolism, Islet Architecture, and Genetic Homogeneity in Imprinting of and Insulin Rhythms in Mouse Islets}, Florida State University Biomathematics Journal Club
% \item[07/13] {\it On Multistability of Delayed Genetic Regulatory Networks with Multivariable Regulation Functions}, Florida State University Biomathematics Journal Club
\item[06/13] {\it Complex Dynamics of Compound Bursting with Burst Episode Composed of Different Bursts}, Florida State University Biomathematics Journal Club
% \item[12/12] {\it Deterministic Models of Gene Networks}, Florida State University Mathematical Biophysics Course
\item[05/08] Undergraduate Thesis Defense, {\it Regular Polytopes and Symmetry}, St. Mary's College of Maryland
\end{CV}

\section{Honors and Awards}
\begin{CV}
\item[06/15] Evelyn and John Baugh Scholarship, Florida State University Department of Mathematics
\item[07/13 - 05/14] Graduate Assistance in Areas of National Need Fellow, U.S. Department of Education
%\item[08/08] Teacher's Certificate, Chicago Public Schools
% \item[05/08] National Collegiate Honors Council Diploma, St. Mary's College of Maryland
% \item[5/07 \& 5/08] William Lowell Putnam Competition Award, St. Mary's College of Maryland Department of Mathematics
% \item[12/06] NCAA Division III Soccer Athlete, St. Mary's College of Maryland
\item[12/05 \& 05/08] Dean's List,  St. Mary's College of Maryland
\item[9/04 - 5/08] Presidential Award,  St. Mary's College of Maryland
%\item[9/04 - 5/08] Kevin E. Reichardt Foundation Scholarship, St. Mary's College of Maryland
% \item[06/04] Eagle Scout, Boy Scouts of America
% \item[05/03 \& 05/04] Magna Cum Laude, National Latin Exam
\end{CV}

\section{Memberships}

\begin{CV}
\item[12/14 - present] Society for Mathematical Biology
\item[12/13 - present] Pi Mu Epsilon, National Honorary Mathematical Society, Florida State University
\item[09/13 - present] Program for Instructional Excellence, Florida State University
\item[04/13 - present] Society for Industrial and Applied Mathematics
% \item[06/08 - present] Mathematical Association of America
% \item[09/04 - 05/08] Mathematics Club, St. Mary's College of Maryland
\end{CV}

% \section{Technical Abilities}
% \begin{CV}
% \item[Computational] C, C++, Fortran, Python, Java, MATLAB, Maple, XPP, AUTO, HTML, Javascript, CSS, Unix OS, \LaTeX
% \item[Language] French, Latin, Twi
% \end{CV}
% 
% \section{References}
% 
% \begin{table}[h]
% \begin{tabular}{@{}lll@{}}
% \textbf{Dr. Richard Bertram} \\
% Director, Biomathematics Program & Phone: & 850-644-7195 (math)\\
% Florida State University &  & 850-644-7632 (IMB)\\
% Department of Mathematics & Fax: & 850-644-4053 \\
% 1017 Academic Way &  Email: & bertram@math.fsu.edu\\
% Room 208 Love Building & & \\
% Tallahassee, FL 32306-4510 & &
% \end{tabular}
% \end{table}
% 
% \begin{table}[h]
% \begin{tabular}{@{}lll@{}}
% \textbf{Dr. Arhtur Sherman} \\
% Director, Laboratory of Biological Modeling & Phone: & 301-496-4325 \\
% National Institutes of Health & Fax: & 301-402-0535 \\
% National Institute of Diabetes and Digestive and Kidney Diseases & Email: & asherman@nih.gov \\
% 12 South Dr MSC 5621  &  &\\
% Bethesda, MD 20892-5621 & & \\
% \end{tabular}
% \end{table}
% 
% \begin{table}[h]
% \begin{tabular}{@{}lll@{}}
% \textbf{Dr. Kyle Gallivan} \\
% Director, Applied Mathematics Program & Phone: & 850-645-0306\\
% Florida State University & Fax: & 850-644-4053\\
% Department of Mathematics & Email: & gallivan@math.fsu.edu \\
% 1017 Academic Way & & \\
% Room 208 Love Building & & \\
% Tallahassee, FL 32306-4510 & &
% \end{tabular}
% \end{table}
% 

\noindent \today



\end{document}

%Tabellen
\begin{table}[htbp] \centering%
\begin{tabular}{lll}\hline\hline
1 & 2 & 3 \\ \hline
1 & \multicolumn{2}{c}{2} \\
\hline
\end{tabular}
\caption{Titel\label{Tabelle: Label}}
\end{table}






